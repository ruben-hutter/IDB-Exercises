
%% General definitions
\documentclass{article} %% Determines the general format.
\usepackage{a4wide} %% paper size: A4.
\usepackage[utf8]{inputenc} %% This file is written in UTF-8.
%% Some editors on Windows cannot save files in UTF-8.
%% If there is a problem with special characters not showing up
%% correctly, try switching "utf8" to "latin1" (ISO 8859-1).
\usepackage[T1]{fontenc} %% Format of hte resulting PDF file.
\usepackage{fancyhdr} %% Package to create a header on each page.
\usepackage{lastpage} %% Used for "Page X of Y" in the header.
											%% For this to work, you have to call pdflatex twice.
\usepackage{enumerate} %% Used to change the style of enumerations (see below).

\usepackage{amssymb} %% Definitions for math symbols.
\usepackage{amsmath} %% Definitions for math symbols.
\usepackage{amsthm}
\usepackage{braket}
\usepackage{graphicx}
\usepackage{float}
\usepackage{hyperref}

\usepackage{tikz}  %% Pagacke to create graphics (graphs, automata, etc.)
\usetikzlibrary{automata} %% Tikz library to draw automata
\usetikzlibrary{arrows}   %% Tikz library for nicer arrow heads


%% Left side of header
\lhead{\course\\\semester\\Exercise \homeworkNumber}
%% Right side of header
\rhead{\authorname\\Page \thepage\ of \pageref{LastPage}}
%% Height of Header
\usepackage[headheight=36pt]{geometry}
%% Page style that uses the header
\pagestyle{fancy}

\newcommand{\authorname}{Nico Bachmann\\Ruben Hutter}
\newcommand{\semester}{Fall Semester 2023}
\newcommand{\course}{Introduction to DB}
\newcommand{\homeworkNumber}{5}


\begin{document}

\section*{Exercise \homeworkNumber.4}

TODO

\section*{Exercise \homeworkNumber.5}

\begin{enumerate}[(1)]
\item The initial INSERTs fail because we are trying to insert into a view, witch is not automatically updatable because it involves a complex expression, the sinus function.

\item We could add the following constraint to the table \textit{measurements} in order to limit the possible x values, so that the arcsin(y) can be calculated to obtain x:\\\\
CREATE TABLE measurements\\
\dots\\
CONSTRAINT validSin CHECK(x IS NOT NULL AND 0 <= x AND x <= 2 * pi())

\item It depends on the DBMS, but in PostgreSQL, views are generally updatable if they meet certain criteria. However, there are limitations, and not all views are automatically updatable. With the fact we are using the sinus function in the view, we will more likely get an error inserting into the view, but if we would have a view which involves a single table and does not contain constructs like aggregate functions or GROUP BY than we would have the entries in the table as well, since a view is just a virtual representation of the data based on the original table.
\end{enumerate}

\section*{Exercise \homeworkNumber.6}

\begin{enumerate}[(1)]
\item \textbf{Graph Database:} A network can be expressed as connected nodes, which is exactly a graph. We probably have to check if two elements are connected and shortest-paths to go from one node to an other. In the case of the railway we would save every station as a node and connect the nodes of the stations connected by rails.

\item \textbf{Column Store:} A bookkeeping system needs to aggregate stored values so basically have fast access to many columns. Let's assume that we have a column \textit{Account Nr.} and a column \textit{Balance CHF}. The entries in classical row stores would store entire tuples on database pages, but if we are only interested in reading the two columns mentioned above, it's better to use a system that stores the columns in physical memory blocks rather than entire tuples.

\item \textbf{Relational Database:} TODO

\item \textbf{Document Store:} TODO

\item \textbf{Key Value Store:} TODO

\item \textbf{Document Store:} TODO
\end{enumerate}

\end{document}
