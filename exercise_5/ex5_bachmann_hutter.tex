
%% General definitions
\documentclass{article} %% Determines the general format.
\usepackage{a4wide} %% paper size: A4.
\usepackage[utf8]{inputenc} %% This file is written in UTF-8.
%% Some editors on Windows cannot save files in UTF-8.
%% If there is a problem with special characters not showing up
%% correctly, try switching "utf8" to "latin1" (ISO 8859-1).
\usepackage[T1]{fontenc} %% Format of hte resulting PDF file.
\usepackage{fancyhdr} %% Package to create a header on each page.
\usepackage{lastpage} %% Used for "Page X of Y" in the header.
											%% For this to work, you have to call pdflatex twice.
\usepackage{enumerate} %% Used to change the style of enumerations (see below).

\usepackage{amssymb} %% Definitions for math symbols.
\usepackage{amsmath} %% Definitions for math symbols.
\usepackage{amsthm}
\usepackage{braket}
\usepackage{graphicx}
\usepackage{float}
\usepackage{hyperref}

\usepackage{tikz}  %% Pagacke to create graphics (graphs, automata, etc.)
\usetikzlibrary{automata} %% Tikz library to draw automata
\usetikzlibrary{arrows}   %% Tikz library for nicer arrow heads


%% Left side of header
\lhead{\course\\\semester\\Exercise \homeworkNumber}
%% Right side of header
\rhead{\authorname\\Page \thepage\ of \pageref{LastPage}}
%% Height of Header
\usepackage[headheight=36pt]{geometry}
%% Page style that uses the header
\pagestyle{fancy}

\newcommand{\authorname}{Ruben Hutter}
\newcommand{\semester}{Fall Semester 2023}
\newcommand{\course}{Informatiklabor}
\newcommand{\homeworkNumber}{12}


\begin{document}

\section*{Einführung} Dieser Bericht berichtet über die Ergebnisse des Benchmarkings welches wir im Informatiklabor durchgeführt haben. Dabei haben wir ein Programm geschrieben welches ein anderes Programm mehrmals mit unterschiedlichen Parametern ausführt und misst, wie lange das andere Programm braucht.

\section*{Polynome}
In der zweiten Sektion werden Polynome eingeführt. Ein Polynom ist ein mathematischer Ausdruck, der aus Variablen (oft bezeichnet als $x$) und Koeffizienten besteht, die durch Addition und Subtraktion verbunden sind.

Die allgemeine Form eines Polynoms ist gegeben durch:
\[ P(x) = a_n x^n + a_{n-1} x^{n-1} + \cdots + a_2 x^2 + a_1 x + a_0 \]
wobei $a_n, a_{n-1}, \ldots, a_1, a_0$ die Koeffizienten sind und $n$ den Grad des Polynoms angibt.

Ein Beispiel für ein Polynom dritten Grades ist:
\[ Q(x) = 2x^3 - 3x^2 + x - 5 \]

\section*{Resulate}
In der dritten Sektion `Resultate` sind die Ergebnisse aus Aufgabe 2. Abbildung~\ref{fig:laufzeitplot} zeigt den Vergleich der Laufzeiten zweier Algorithmen in Abhängigkeit von verschiedenen Parametern.

\begin{figure}[h]
    \centering
    \includegraphics[width=0.7\textwidth]{scatterplot_with_regression.png}
    \caption{Vergleich der Laufzeiten für Algorithmus 1 und Algorithmus 2. Die Datenpunkte sind nach dem jeweiligen Algorithmus gruppiert und bieten einen direkten Vergleich der Ausführungszeiten.}
    \label{fig:laufzeitplot}
\end{figure}

\end{document}
